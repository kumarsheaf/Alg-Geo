\documentclass[12pt]{article}

% Packages
\usepackage[utf8]{inputenc}
\usepackage[T1]{fontenc}
\usepackage{amsmath, amssymb, amsthm}
\usepackage{mathtools}
\usepackage{geometry}
\usepackage{enumitem}
\usepackage{tikz}
\usepackage{tcolorbox}
\usepackage{hyperref}
\usepackage{fancyhdr}
\usepackage{physics}
\usepackage{bbm}
\usepackage{mathrsfs}
\usepackage{graphicx}
\usepackage{xcolor}

% Page geometry
\geometry{a4paper, margin=1in}

% Header
\pagestyle{fancy}
\fancyhf{}

\cfoot{\thepage}

% tcolorbox settings
\tcbset{
  colback=white,
  colframe=black,
  fonttitle=\bfseries,
  coltitle=black,
  sharp corners,
  boxrule=0.8pt,
  arc=3pt,
  left=5pt,
  right=5pt,
  top=5pt,
  bottom=5pt,
  enhanced,
  breakable
}

% Custom theorem boxes
% Box definitions with custom title argument
\newtcolorbox{theoremBox}[1]{
  colback=blue!5!white, colframe=blue!75!black,
  title={#1}, fonttitle=\bfseries, breakable
}

\newtcolorbox{definitionBox}[1]{
  colback=green!5!white, colframe=green!60!black,
  title={#1}, fonttitle=\bfseries, breakable
}

\newtcolorbox{lemmaBox}[1]{
  colback=orange!5!white, colframe=orange!85!black,
  title={#1}, fonttitle=\bfseries, breakable
}

\newtcolorbox{corollaryBox}[1]{
  colback=purple!5!white, colframe=purple!80!black,
  title={#1}, fonttitle=\bfseries, breakable
}

\newtcolorbox{exampleBox}[1]{
  colback=gray!10!white, colframe=gray!60!black,
  title={#1}, fonttitle=\bfseries, breakable
}

\newtcolorbox{remarkBox}[1]{
  colback=yellow!10!white, colframe=yellow!50!black,
  title={#1}, fonttitle=\bfseries, breakable
}

% Shortcuts
\newcommand{\R}{\mathbb{R}}
\newcommand{\C}{\mathbb{C}}
\newcommand{\Z}{\mathbb{Z}}
\newcommand{\N}{\mathbb{N}}
\newcommand{\Q}{\mathbb{Q}}
\newcommand{\de}{\,\mathrm{d}}
\newcommand{\del}{\partial}
\newcommand{\eps}{\varepsilon}

% Title
\title{\textbf{Algebraic Geometry}}
\author{Kumar Satyadarshi}
\date{\today}

\begin{document}

\maketitle
\tableofcontents
\newpage

\section{Affine Varieties}
\rhead{Affine Varietries}
\lhead{Algebraic Geometry}
In this section we will always work over a fixed algebraically closed field \textit{k} .
\begin{definitionBox}{Definition 1.1}
The set of all n-tuples of \textit{k} is know as affine n-space over \textit{k}. It is usually denoted by $\mathbf{A}^{n}_{k}$ or simply $\mathbf{A}^{n}$.
\end{definitionBox}
\noindent An element P $\in \mathbf{A}^{n}$ will be called a \textit{point} and if P =  $(a_{1},...,a_{n})$ with $a_{i} \in \textit{k}$, then $a_{i}$ will be called the \textit{coordinates} of P. 
\begin{remarkBox}
\ Let A = \textit{k}[$x_{1}, ... ,x_{n}$] and $f \in $ A, then we can talk about zeros of $f$, namely $$ Z(f) = \{P \in \mathbf{A}^{n} \;|\; f(P) = 0\}$$
More generally, if $T \subset A$ then we will define zero set of T to be common zeros of all elements of T, namely $$ Z(T) = \{ P \in \mathbf{A}^{n} | f(P) = 0 \text{ for all f in T }\}  $$
\end{remarkBox}
\noindent It is very easy to see that if $I$ is the ideal of A generated by T, then Z($I$) = Z(T). By Hilbert basis theorem, A is a noetherian ring. So, any ideal $I$ has finite set of generators $f_{1},...f_{r}$. Thus, Z(T) can expressed as the common zeros of the finite set of polynomials $f_{1},...f_{r}.$
\begin{definitionBox}{Definition 1.2}
    A subset Y of $\mathbf{A}^{n}$ is an \textbf{algebraic set} if there exists a subset $T \subset A$ such that Y = Z(T).
\end{definitionBox} \noindent
\textbf{Examples .} Here are some examples of algebraic sets.
\begin{enumerate}
    \item Affine n-space: $\mathbf{A}^{n} = Z(0)$.
    \item The empty set : $ \phi = Z(1).$
    \item Any single point: $(a_{1},...,a_{n}) = Z(x_{1} - a_{1},...,x_{n} - a_{n})$
    \item Linear subspaces of $\mathbf{A}^{n}$. We can always write any linear subspace as intersection of planes.
\end{enumerate}
\end{document}
